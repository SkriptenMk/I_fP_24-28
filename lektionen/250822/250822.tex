\documentclass[a4paper]{scrreprt}
\usepackage[german]{babel}
\usepackage[utf8]{inputenc}
\usepackage{array}
\usepackage{booktabs}
\usepackage{longtable}
\usepackage{ragged2e}
\usepackage{enumitem}
\setlist[itemize]{noitemsep}

\begin{document}
\section*{2fP, 22. August 2025, IP-Adressierung}
\begin{longtable}{p{1.5cm}>{\RaggedRight}p{7.5cm}p{2.5cm}}
    \toprule
    \emph{Zeit}&\emph{Inhalt}&\emph{Methode}\\
    \midrule
    \endhead

    \midrule
    \multicolumn{3}{c}{\begin{tiny}\textit{to be continued}\end{tiny}}\\
    \midrule
    \endfoot

    \bottomrule
    \endlastfoot

    1325&Semestertermine&\\ [5pt]

    1330&Was geschieht beim Aufruf einer Internetadresse?&Lehrgespräch\\
        &Übersetzung Domain Name in IP-Adresse&\\
        &nslookup www.nzz.ch (auf dem default DNS Server)&\\
        &nslookup www.nzz.ch 1.1.1.1&\\
        &nslookup www.nzz.ch 9.9.9.9&\\
        &Vergleich der Resultate&\\ [5pt]

    1345&IPv4 im Vergleich zu IPv6&Lehrgespräch\\
        &Darstellung und Anzahl Adressen&\\ [5pt]

    1355&Lokale IP-Adresse&Lehrgespräch\\
        &ipconfig&\\
        &Besprechung der Ausgabe inklusive der Subnetmaske&\\ [5pt]

    1420&Externe IP-Adresse&Lehrgespräch\\
        &https://whatismyip.org&\\ [5pt]

    1425&NAT&Lehrgespräch\\ [5pt]
        &Direkter Aufruf von Google via IP Adresse unter Angabe des
        Portes 80&\\ [5pt]

    1435&OSI bzw. TCP/IP Layer Modell&Lehrvortrag\\ [5pt]

    1445&ev. Installation von Wireshark&Lehrvortrag\\




\end{longtable}
\end{document}
